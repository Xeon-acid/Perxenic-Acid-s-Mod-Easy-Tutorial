\section{Engineer}
        Only self-using notes.
        \subsection{Linear Algebra Basic}
            \par Reviewing about some basic linear algebra.
            \begin{definition}
                An \(n\)--tuple \((x_1, x_2, \dots, x_n)\) denotes recursively \(((x_1, \dots, x_{n-1}), x_n)\), and a 2-tuple \((x,y)\) denotes set \(\{x, \{x,y\}\}\)\ (by Kuratowski).
            \end{definition}
            A vector can not be a tuple, and a tuple can not be a vector.
            \begin{definition}
                A \textbf{Vector Space} denotes a 4--tuple \((\mathbb{F}, V; \mathbf{0}_V; +_V, \cdot)\) that:
                    \begin{itemize}
                        \item \(\mathbb{F}\) is a field \((F; 0_F, 1_F; +_F, \cdot_F)\);
                        \item \(V\) is one non-empty set;
                        \item \(+_V\) is an operator on \(V\) that \((V; \mathbf{1}_V; +_V)\) is an Abelian group;
                        \item \(\cdot\) is an operator between \(\mathbb F\) and \(V\) that:
                            \begin{itemize}
                                \item \(\forall a\in\mathbb{F},\forall v\in V, av\in V\);
                                \item \(\forall v\in V, 1_Fv=v\);
                                \item \(\forall a\in\mathbb{F},\forall v,w\in V, a(v+_Vw)=av+_Vaw\);
                                \item \(\forall a,b\in\mathbb{F},\forall v\in V, (a+_Fb)v=av+_Fbv\).
                            \end{itemize}
                    \end{itemize}
            \end{definition}
        \subsection{Transformation}
        \subsection{Rasterization}
        \subsection{Shader}
        \subsection{Buffer}
        