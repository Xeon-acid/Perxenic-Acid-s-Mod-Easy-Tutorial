\section{Player}
        无论读者希望制作 MOD 还是仅仅希望游玩 MOD, 都应当基本了解本部分内容.
        \subsection{准备工作}
            \par 首先需要一款基于 Unity 的游戏, 据笔者所知, 原神、崩铁、绝区零、崩三、来自星尘这些游戏均采用 Unity 3D 实现, 这些游戏基本都可以使用 3DMigoto 使用 MOD. 对于其他游戏, 有可能并非使用 Unity 实现, 也许使用的是 Unreal Engine 或其他引擎, 这些游戏可以通过其他工具使用 MOD, 但笔者并不了解, 在此不详述了.
            \par 接下来, 需要拿到 3DMigoto. 根据你需要的游戏不同, 3DM 有不同的特化版本, 如 Genshin Impact Model Importer (i.e. GIMI), SRMI, ZZMI, HIMI 等\footnote{鸣潮的 MOD 虽然通过完全不同的方式实现, 但其 MOD 加载器采用相近的形式命名为 WWMI.}, 有些游戏没有这样的特化版本, 通常使用 GIMI 即可. 可以在 GitHub 下载这些工具, 比如 GIMI 可以在\href{https://github.com/SilentNightSound/GI-Model-Importer/releases/tag/v7.0}{这里}点击进入下载界面, 这里会提供两个版本的 \texttt{.zip} 文件, 一个是开发者版本, 一个是游玩者版本, 通常推荐直接下载前者, 后者所谓的精简一些, 牺牲了大量功能. 其他游戏在使用 MOD 时所需的 MOD 加载器, 有时可以直接使用原神 MOD 加载器 GIMI, 也可以自行查找 GitHub 库, 恕不赘述.
            \par (强烈推荐) 使用一个更趁手的文本编辑器, 比如 Visual Studio Code, 可以点击\href{https://vscode.download.prss.microsoft.com/dbazure/download/stable/384ff7382de624fb94dbaf6da11977bba1ecd427/VSCodeUserSetup-x64-1.94.2.exe}{这里}以\textbf{直接}下载. 有些读者或许习惯使用 NotePad (记事本) 编辑文本, 请改掉这个\textbf{陋习}.
        \subsection{配置}
            \par 需要相当提前地指出: 3DMigoto 工具能够实现 MOD 热重载, 这也意味着增删补改 MOD 文件不需要重启游戏\footnote{绝大多数情况如此.}.
            \par 下载解压好 GIMI 后, 于其根目录打开 \texttt{d3dx.ini}, 定位到其第 539 行, 替换内容为
            \begin{lstlisting}[basicstyle=\ttfamily]
        target = D:\Genshin Impact\Genshin Impact Game\YuanShen.exe
            \end{lstlisting}
            后面跟的地址并不固定, 依读者设备上的 \texttt{YuanShen.exe} 所在路径而定. \keys{Ctrl + S}保存.
            \par 接下来测试是否成功配置. 在 GIMI 根目录启动可执行文件 \texttt{3DMigoto Loader.exe}, 待控制台窗口出现并显示
            \begin{lstlisting}[basicstyle=\ttfamily]
    ------------------------------- 3DMigoto GIMI Loader ------------------------------

    d3d11.dll description: "3Dmigoto - d3d11.dll"
    3DMigoto Version 1.3.16
    Loaded G:\3dmigoto\d3d11.dll

    3DMigoto ready - Now run the game.            
            \end{lstlisting}
            后, 启动原神 (通过官方/第三方启动器启动或直接启动游戏本体均不影响.), 在进入 ``原神'' 启动 Logo 后, 在确保 \keys{NumLock} 已启用的前提下点按 \keys{Num0}\footnote{如果不知道这是什么的话: 请点按小键盘区的数字键 0.}, 如果屏幕上下出现绿色文本, 则说明配置成功. 再次点按 \keys{Num0} 关闭这些文本. 如果并未出现, 说明使用的是游玩者版本, 或者并未正确配置, 可以在下一步继续审查.
            \par 进行下一步之前\textbf{无需}关闭游戏.

        \subsection{使用已有的 MOD}
            \par 首先请\href{https://filecache33.gamebanana.com/mods/lumine_lantern_rite.zip}{点击}以下载笔者的拙作, 这是荧妹的一个 MOD, 基于他人的 MOD 进行了简单的重着色处理. 如果第一次使用该网站, 或许需要注册一个账号.
            \par GIMI 根目录下应当存在名为 \texttt{Mods} 的子目录, 打开该目录, 将下载的压缩文件解压缩至该处, 然后回到游戏, 点按 \keys{F10} 以重载 MOD. 如果发现荧妹的模型发生修改, 说明 MOD 成功使用. 在 GameBanana 上下载的 MOD, 有时并非置于 \texttt{Mods} 目录, 而是 \texttt{ShaderFixed} 或 \texttt{ShaderCache}, 一般在 MOD 简介中有特别说明, 留意即可.

            \subsubsection{关于 MOD 管理}
                \par 3DMigoto 的功能非常强大, 基本可以读取任意层级子目录的 MOD, 甚至所有 MOD 混置于一处也能够正确识别 (不可有重名文件). 尽管如此, 仍然推荐细致分类所有 MOD, 因为对同一个对象的多个 MOD 在同时启用时, 3DM 会老老实实同时加载两份 MOD, 这几乎一定会导致游戏中出现极其抽象的模型碎块.
                \par 3DM 读取 MOD 依靠的是 \texttt{.ini} 文件的内容, 可以更改其后缀名以禁用 MOD. 一个更规范的方式是, 添加字符串 \texttt{DISABLED} 于任何一个需要禁用的文件或目录的前缀名之前, 比如需要禁用 \texttt{Lumine.ini}, 则将其改名为 \texttt{{\color{blue!50}DISABLED}Lumine.ini} 或 \texttt{Lumine.ini{\color{blue!50};}}, 或者将其父级目录 \texttt{LumineMod} (或者其他任何你喜欢的名字, 3DM 不会理会这些目录名.) 重命名为 \texttt{{\color{blue!50}DISABLED}LumineMod}.
                \par 值得指出, 存在较为方便的 MOD 管理程序, 但笔者不感兴趣, 并不了解.
        \subsection{在哪下载 MOD?}
            \par 前文提到的 \href{https://gamebanana.com/}{GameBanana}. NSFW 的 MOD 需要注册账号并在设置中取消 NSFW MOD 隐藏, 不过该网站不允许任何有关萝莉的 NSFW MOD.